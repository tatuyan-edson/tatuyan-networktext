# 第2章 デジタルデータの表現

前章では、電気を用いない通信を俯瞰し、通信と記録が密接な関係にあることや、記録するときには記録するべき要素の取捨選択があることを見てきました。また、情報の完全性・可用性・機密性を始めとして通信に対し要求あるいは期待されるものも顕れていました。

人間が直接運ぶにせよ光や音を使うにせよ電気通信にせよ、記録されたものを受け渡しするのが通信です。この章と次の章では、電気を用いた通信のために、電気を用いて記録するとはどういうことなのか、情報理論という観点から見ていきます。まずは、離散的な「デジタルデータ」の表現から見ていきましょう。

## デジタルデータの特徴と二元状態

デジタルデータは、任意の数のデータを表すことさえできればデジタルのデータを表すことができます。
これが離散的な表現をしたときのデジタルデータの特徴です。

この任意の数というのは、コンピュータは0と1のふたつ、即ち二進数です。
なぜそのような表し方でいいのかと言うと、コンピュータに電気信号を流すか否か、というはっきり分かれたものにすることによって定量的にものを表すことができ、色々な観点で都合がいいからです。
この状態のことを二元状態と呼びます。

## 情報の符号化

情報の符号化とは、コンピュータの1と0に情報を置き換えることです。
では実際に情報の符号化はどのようにして行われるのでしょうか。

### 数値や文字の符号化:情報の表現

数値や文字を符号化する際使われる二進数は、bit(binary digit - 二進数の意)と呼ばれます。
これを用いて符号化を行うのですが、例えば整数と非負整数の違いの表現は2の補数などを用いたり、文字はASCIIなどの文字コードをbitに変換して表現する、などの工夫が符号化する際に行われます。

## 符号化効率の改善:エントロピー符号

しかし符号化に当たって、全て正直に符号化するのはしんどいし解析が面倒、そして符号長が長くなってしまうということが起きると考えられます。

ではどうすればいいか、ということなのですが、
よく使う記号を短い符号にすれば必然的に短くなるのではないかと考えられて生み出されたのがエントロピー符号です。

### 符号の具体例:モールス符号

モールス符号は長さが一定でない符号化の代表的な例として挙げられます。

アルファベットで実際に使う頻度が最も高いものはAやE、Nなどです。
モールス符号ではこれらの文字は比較的短い符号に設定されています。

しかしモールス符号では、大文字と小文字の区別がないなど、やはり情報の欠落が見られます。換言すれば、アルファベットのみによる情報しか送ることができないのです。

### 基礎的なエントロピー符号アルゴリズム:ハフマン符号

基礎的なエントロピー符号アルゴリズムとして、ハフマン符号化があります。

概要としては、トップダウンな形を例に挙げて説明すると、一番出る確率が高いもものを1、二番目に高いものは01、三番目に高いものは001...といった形で符号化すると決めると、各データの出る確率が十分に偏っている場合、出る確率が高い順で符号長は短くなるので、結果的に全体の平均符号長は短くなります。

ハフマン符号化で生成される符号の平均符号長は、各値の出現比率の期待値で求められます。

### 演習問題:ハフマン符号化
データの個数と、各データの出現比率が与えられるとき、そのハフマン符号化の例と、その際の平均符号長を出力するプログラムを作成してみよう。

例:データが4種類

* データ1:4
* データ2:2
* データ3:1
* データ4:1

## より知りたい時のために
このあたりの内容は、情報理論の書籍に詳しい。

また、具体的な符号化のアルゴリズムとして、有名なものを挙げておく。より学びたい時の参考キーワードとして利用されたい。

* シャノン符号
* 算術符号・レンジ符号
* ガンマ符号・デルタ符号


