電気を用いない通信から見てきた通り、通信と記録は密接な関係にある。また、記録の際には記録するべき要素の取捨選択が行われる。人間が直接運ぶにせよ光や音を使うにせよ電気通信にせよ、記録されたものを受け渡しするのが通信であるということに立ち戻れば、記録と通信は表裏一体の関係にあることは明らかであろう。

本章では、離散的な「デジタルデータ」をいかにして表現・記録するかという点について学んでいく。


%--- Tomohiro memo---
# 第6章 伝送路の方式と性質

ここまでの章で、記録することと、それを伝送するのに必要な要件を確認してきました。この章では、それを用いる側、伝送路・通信路の方式や性質について見ていきましょう。

## 通信方式:片方向・双方向
通信方式には片方向と双方向の方式がありますが、この2つを糸電話を例に挙げて説明していきます。

片方向通信は、糸電話が一本で、なおかつ聞く側(受信側)と話す側(送信側)が固定されている通信方式のことを言います。

一方、双方向通信は受信側と送信側が場合に応じて入れ替わる方式ですが、これも2種類に分けることができ、一つは聞くチャネルと話すチャネルと切り替えながら一本の糸電話で通信する半二重通信、もう一つは糸電話を2つ用意し、片方は送信機、片方は受信機として使用する全二重通信といったものです。

## 信号伝送方式
信号伝送方式にはベースバンド伝送方式と帯域伝送方式の2種類があります。

帯域伝送方式は搬送波にデータを乗せて伝送する方式で、振幅変調、位相変調、周波数変調、振幅位相変調といったものがこれの例に挙げられます。  
この中にある振幅位相変調は、二進数の桁を多くとることができるようになり、一度にたくさんのデータ(大きな値)を搬送波に載せることができるようになるといった利点があります。

ベースバンド伝送方式は0と1の電圧の差でデータを伝送する方式で、これにもいくつかの種類があります。 
例えば、

・return to 0方式(RZ方式)…一回に送るデータのうち半分を0にする方式。  
・non return to 0方式(NRZ方式)…上とは逆に、一回に送るデータの半分を1にする方式。  
・両極方式…ある+1という電圧から-1という電圧までを使う方式。仮に-1を1とすると、+1が0とされる。  
・単極方式…+1と0(または-1と0)の電圧を使う方式。

という4つの方式がありますが、これらは2つを組み合わせて使います。(ex.両極RZ方式、単極NRZ方式など)

ほかにも電圧の値を絶対値で取り、+1か-1かは前回取った方とは違う方という風に判定するバイポーラ方式、ビットの真ん中で電圧が変わったのを判定して、値が上昇したら1、下降したら0と判定するマンチェスター方式などがあります。

## 伝送速度
伝送速度は、一般的に「1秒間にどれだけのビットを送ることができるか」を数値で表したbps(bit per second)を用いて表します。

もう一つ、変調伝送速度というものもあり、これは「どれだけのデータを1秒間の搬送波に載せられるか」を数値で表すもので、単位はボー(Baud)を用います。

### 演習問題
変調1回につき2^k 通りの情報を表すことができ、変調に用いる搬送波の周波数がf Hzであるような波があるとき、その変調速度(Baud)と伝送速度(bps)を出力するプログラムを作成してみましょう。但し、k,fは正の整数値で、入力により与えられるものとします。

## 通信品質
通信品質には3つの基準があります。

1つ目は接続基準で、接続するまでに何秒かかるかを、基本的に秒数で計測した基準です。

2つ目は安定基準(=安定率)で、使えた時間/使いたい時間をパーセンテージで表した基準です。

3つ目は伝送基準で、伝送のときの欠損や雑音、値の変化などの少なさをデジタルでは符号誤り率、アナログではS/N比(Single to Noiselation)を用いて表した基準です。

%--- Naoppy memo---

# 第6章 伝送路の方式と性質

ここまでの章で、記録することと、それを伝送するのに必要な要件を確認してきました。この章では、それを用いる側、伝送路・通信路の方式や性質について見ていきましょう。

今までは表現方法、今回は伝える道の話

伝送路のパラメーター:太さ、長さ、スピード、方向、品質


## 通信方式:片方向・双方向

1対1
一つの糸電話ですると...聞く側と話す側にわかれないといけない→単方向通信(ex. テレビ、ビデオ出力端子)

聞く側と話す側をうまく切り替えながら通信すること→半二重通信

常に聞く線と常に話す線の2つを持って通信すること→全二重通信

LANケーブルは送信線が2本、受信線が2本、何もしない線が4本入っている

スイッチなどでは片方の受信ともう片方の送信のポートを繋いでいる。なのでそのまま愚直に繋いでOK。ストレートケーブルという。

PC to PCやスイッチ to スイッチの場合、内部で逆にする必要がある。これがクロスケーブルという。

## 信号伝送方式

電気や光、電波など。

伝送方式にはベースバンド伝送方式と帯域伝送方式がある。

### 帯域伝送方式
搬送波にデータをのせて送る。

変調によって搬送波に載せる

直交振幅変調など...x、yの2つのパラメータを変えると、(xのパターン数×yのパターン)数の状態を表現できる。

###ベースバンド伝送方式
0と1の電圧で送る伝送方式。

return to zero方式...一度におくるものの半分は0で半分は1

not return to zero方式...ずっと1が送られる

両極方式...+1から-1までの電圧を使う、端が0と1

単極方式...+ or - 1から0までの電圧を使う、端が0と1

バイポーラー方式...+1と-1を使うが、+1と-1は両方1を表し、0は0を表す。前回1で1を表現したのなら、次は-1で1を表現しなければならない。

マンチェスター方式...あるbitを示す時間の間が-1 to 1(増えた)なら1 to -1(減った)なら0になる。
現代のインターネットはこれ


## 伝送速度
データ伝送速度...Mbpsなど

変調速度...どれだけのデータを1秒の搬送波に載せられるかを表す...単位はbaud

### 演習問題
変調1回につき2^k 通りの情報を表すことができ、変調に用いる搬送波の周波数がf Hzであるような波があるとき、その変調速度(Baud)と伝送速度(bps)を出力するプログラムを作成してみましょう。但し、k,fは正の整数値で、入力により与えられるものとします。

## 通信品質
劣化しない、欠損が少ない、混信しない、断信しない

3つの基準
接続基準、安定基準、伝送基準

接続基準...接続に何秒かかるか

安定基準...稼働率
稼働率=使えた時間 / 使いたい時間

伝送基準...伝送の誤りがどの程度か
符号誤り率(何%の確率で01を誤るか
信号の量に対してノイズがどれだけ入るか(S/N比
%--- memo ここまで---

\section{}

\section*{演習問題}
\begin{problems}
\item ほげ
\end{problems}
