\section{ビット同期}
\section{ブロック同期}



%--- Tomohiro memo---
# 第4章 電気信号の伝送

ここまでに、情報をどのようにして記録するかを見てきました。こうして符号化されたデータは2進数の0と1に対応するため、2元的な電気信号を生成し、これを送ることができれば電気通信となりそうです。もちろんその理解で良いのですが、単に電気信号を送るだけでは不便な部分も有ります。

この章では、電気信号そのものを送ることについて見ていきましょう。

いままでは0と1のデータがどこで始まってどこで終わるかがわかっていたから理解できた。しかし実際、その開始と終了は分からない。

こういう場合どうすればいい(どうやって同期すればいい)のか?

#電気信号の同期方式

##bit同期について

bit同期とは、理想的な電圧がかかっているとして、それをどこで区切って置き換えるか、一つの電気信号の0と1をどうやって定義するかを決めるものである。

##同期パルス方式

同期用の一定のパルスを決め(同期パルスという)、そのとおりに電気信号を送ることで一定区間内で0と1を解釈するという同期方式。

##調歩同期方式

電気信号の解釈は、極論最初の合図と最後の合図がわかれば一応することができる。
この最初の合図をスタートビット、終わりの合図をエンドビットといい、これを基準にして電気信号を解釈する同期方法が調歩同期方式である。

#データの区切りについて

電気信号自体の同期は出来ても、「関係のない信号とデータの境目」は意外とわからない。調歩同期方式だとデータの塊の判別はできるけどbit同期だとそれが出来ない。

ではデータの塊はどうすれば同期することができるのか?

##ブロック同期方式

データの塊も、どこから始まっているかという合図があれば同期することができる。
ブロック同期方式は、「データがどこからか」ということを解釈する合図を作り、それに合わせて同期するという同期方式である。

##フラグ同期方式

考えを変えてみると、ある特定の電気信号を決めて、それ「以外」をデータと解釈するという方法も考えられる。

その方法がフラグ同期方式で、決められたbit列をフラグパターンとし、そのフラグパターン「以外」をデータと解釈して取り出す同期方法である。

##キャラクタ同期方式

始めることを意味する特定の8文字の文字列を決め(これをSYN[シン]という)、そのSYNがきたらその後がデータであると解釈する同期方式。

###調歩同期方式との違いは?

調歩同期方式とキャラクタ同期方式の違いは、前者は「データ」を、後者は「電気信号」を解釈するという解釈するものの違いである。

### 演習問題:bit列のフラグ同期
bit列が与えられます。これは、伝送されてきたビット列とします。このとき、フラグパターンを01111110とするフラグ同期方式で同期を取り、そこから8bit毎に区切って得られるデータをchar型の文字列として出力するプログラムを作成してください。

ここまでで1回

---

ここからで1回

## 電気信号の伝送方式

## 変調と復調

我々は搬送波というもので今までは考えて来ましたが、考え直してみると通常使われている電気は交流です。
交流ということは波を送っているということであり、もしその波を変えることが出来たならその差分部分に意味(データ)を入れて送ることができると推測できます。

この波を変えることを変調といい、逆に変調した波を元の波にもどしてデータを取り出す行為を複調といいます。

ここではアナログ変調で考えますが、変調の種類は大きく分けて振幅変調(Amplitude Modulation - AM)、周波数変調(Frequency Modulation - FM)、位相変調(Phase Modulation - PM)の3つがあります。

###振幅変調
振幅変調は、搬送波に対して振幅を変える変調の方法です。

具体的にいうと変調波と長高周波な搬送波を掛け合わせて振幅変調波を作り出しています。

これが使われている例としてAMラジオが挙げられますが、AMラジオは周波数が多少狂っていても(ノイズは混じりますが)ほぼきちんと聞くことが出来ます。
これはなぜかというと、振幅変調は複調するときに振幅の最大値の点を取りますが、その最大値から少しずれても複調したときに波の概形があまり変わらないからです。

###周波数変調

周波数変調は、搬送波の周波数部分を変える変調方法です。

信号波のy座標が高ければ、送信波の周波数が高くなります。
また、性質上振幅は変わらず一定なのでこれが使われている顕著な例であるFMラジオでは雑音が入りにくいです。

###位相変調

位相変調は搬送波の位相を変える変調方法です。

(これ以上書くことが思いつかない)

## 電気信号の同時伝送:多重化方式

多重化方式は大きく分けて周波数分割多重化方式(FDM)、時分割多重化方式(TDM)の二つがあります。

###周波数分割多重化方式(FDM)

この方式は、位相をちょうど半周期ずらした波を用意し(+と-で違うものを送るみたいな感じ)、干渉しない複数の周波数を足し合わせて一気にデータを送信するものです。

人間の声で例えると、複数の人から一気に話を聞いたときにこの声の高さ(周波数)は高いからこの人、こっちは低いからこの人...みたいな感じで声の高さで人を判別する、といったイメージです。

###時分割多重化方式(TDM)

この方式は、波に連続性はいらないという特性を利用し、ピークの位置をずらして波を送っていく、というものです。

先ほど同様人間の声でたとえると、さっきとは違って時間で聞く人を分割し、この時間に聞いているのはこの人の話、こっちの時間に聞いているのはこの人の話...といった感じで時間で区切って話を聞いていくことで複数の人物の話を聞くことができる、と言ったイメージです。

### 演習問題:ローパスフィルタ・ハイパスフィルタ
前章の演習問題では、DFTによりスペクトルを計算するプログラムを作成しました。この計算結果に対し、ローパスフィルタ・ハイパスフィルタをかけるプログラムを作成してみましょう。これを用いた後のデータを復号化し、元のデータと比べてみましょう。

## パルス符号変調

パルス符号変調は、デジタルデータのパルスをアナログな搬送波にして、搬送波に乗せて送る変調方式のことです。
%--- memo ここまで---

\section{}

\section*{演習問題}
\begin{problems}
\item ほげ
\end{problems}
