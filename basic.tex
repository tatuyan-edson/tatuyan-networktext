\documentclass[a4j,12pt]{jreport}
\makeatletter
\usepackage{url}
\usepackage{jtygm}
\usepackage{ascmac}
\usepackage[dvipdfmx]{graphicx}
\usepackage{amsmath,amssymb}
\usepackage{multicol}
\usepackage{listings}
\usepackage{makeidx}
\usepackage{ccaption}
\usepackage{here}
\usepackage{subfigure}
\usepackage{enumerate}
\usepackage[dvipdfmx]{color}
\usepackage{fancybox}
\usepackage{geometry}
\usepackage{ascmac}
\usepackage{titletoc}
\usepackage{titlesec}
\usepackage{framed}
\usepackage{longtable}
\usepackage{ulem}
\usepackage{anyfontsize}
\usepackage{wrapfig}
\usepackage{bm}

\renewcommand{\lstlistingname}{リスト}

\newcommand{\minisec}[1]{
\subsubsection{【#1】}
}

\geometry{
body={384pt,574pt},
hmargin={2.0cm,2.0cm},
vmargin={2.5cm,2.0cm},
bindingoffset=0.5cm,
twoside
}

% 付録の始まり
\newcommand{\beginappendix}{
  % 章番号の書式変更
  \setcounter{chapter}{0}
  \renewcommand{\prechaptername}{付録}
  \renewcommand{\postchaptername}{} 
  \renewcommand{\thechapter}{\@Alph\c@chapter}
  \renewcommand{\thesection}{\@Alph\c@chapter.\@arabic\c@section}
  \renewcommand{\thesubsection}{\@Alph\c@chapter.\@arabic\c@section.\@arabic\c@subsection}
}

%図番号の書式変更
  \renewcommand{\thefigure}{%
  \thechapter.\arabic{figure}}
  \@addtoreset{figure}{chapter}
\makeatother


\newcommand{\ruby}[2]{%
\leavevmode
\setbox0=\hbox{#1}%
\setbox1=\hbox{\tiny #2}%
\ifdim\wd0>\wd1 \dimen0=\wd0 \else \dimen0=\wd1 \fi
\hbox{%
\kanjiskip=0pt plus 2fil
\xkanjiskip=0pt plus 2fil
\vbox{%
\hbox to \dimen0{%
\tiny \hfil#2\hfil}%
\nointerlineskip
\hbox to \dimen0{\mathstrut\hfil#1\hfil}}}}

\makeatletter
\lstset{% 
language={C}, 
frame=trbl,% 
basicstyle={\small},% 
identifierstyle={\small},% 
commentstyle={\small\ttfamily},% 
keywordstyle={\small\bfseries},% 
ndkeywordstyle={\small},% 
stringstyle={\small\ttfamily}, 
tabsize=2,
breaklines=true, 
frame=none,
columns=[l]{fullflexible},% 
numbers=left,% 
xrightmargin=0zw,% 
xleftmargin=3zw,% 
numberstyle={\scriptsize},% 
stepnumber=1, 
numbersep=1zw,% 
backgroundcolor={\color[gray]{.90}},
} 

%footnoteにおいてverbを有効にする設定
\long\def\@footnotetext{%
  \insert\footins\bgroup
    \normalfont\footnotesize
    \interlinepenalty\interfootnotelinepenalty
    \splittopskip\footnotesep
    \splitmaxdepth \dp\strutbox \floatingpenalty \@MM
    \hsize\columnwidth \@parboxrestore
    \protected@edef\@currentlabel{%
       \csname p@footnote\endcsname\@thefnmark
    }%
    \color@begingroup
      \@makefntext{%
        \rule\z@\footnotesep\ignorespaces}%
      \futurelet\next\fo@t}
\def\fo@t{\ifcat\bgroup\noexpand\next \let\next\f@@t
                                \else \let\next\f@t\fi \next}
\def\fo@t{\bgroup\aftergroup\@foot\let\next}
\def\f@t#1{#1\@foot}
\def\@foot{\@finalstrut\strutbox\color@endgroup\egroup}

\renewcommand{\seename}{→}
\makeatother

%-----目次及び見出し-----
\contentsmargin{0pt}
  \titlecontents{chapter}[5.6pc]
  {\addvspace{10pt}\bfseries}
  {\contentslabel[\thecontentslabel]{4.6pc}}
  {}
  {\hfill\normalfont\thecontentspage}

  \titlecontents{section}[4.8pc]
  {\addvspace{4pt}\bfseries}
  {\contentslabel[\thecontentslabel]{2.8pc}}
  {}
  {\dotfill\thecontentspage}

  \titlecontents{subsection}[6.4pc]
  {\addvspace{2pt} }
  {\contentslabel[\thecontentslabel]{3.4pc}}
  {}
  {\dotfill\normalfont \thecontentspage }

 \titleformat{\chapter}[frame]
  {\small}
  {\filright
   \large
  第 \thechapter 章}
  {8pt}
  {\LARGE\bfseries\filcenter}

\titleformat{\section}[hang]{\large\bfseries}
{\colorbox{black}{\color{white}\thesection}}{12pt}{}%
[{\titlerule[1pt]}]

\titleformat{\subsection}[block]
{\normalfont\bfseries}{\fbox{\itshape\thesubsection}}{1em}{}
[{\titlerule[0.5pt]}]

 \titlespacing{\chapter}{0pt}{4pt plus 2pt minus 1pt}{2pt plus 2pt minus 2pt}
 \titlespacing{\section}{0pt}{0pt plus 0pt minus 0pt}{0pt plus 0pt minus 0pt}
 \titlespacing{\subsection}{0pt}{0pt plus 0pt minus 0pt}{0pt plus 0pt minus 0pt}
\newenvironment{code}{
\VerbatimEnvironment
\begin{quote}
\begin{Verbatim}	
}
{
\end{Verbatim}
\end{quote}
}

\newenvironment{problems}
{
  \renewcommand\labelenumi{\doublebox{\arabic{enumi}}}
  \begin{enumerate}
}{
  \end{enumerate}
  \renewcommand\labelenumi{\arabic{enumi}.}
}


\makeindex

\begin{document}
\makeatletter
\renewcommand{\thelstlisting}{\thechapter.\@arabic\c@lstlisting}
\makeatother

\title{{\normalsize 人の言を通す}\\ 通信ひとわたり}
\author{達哉ん}
\date{2018年6月執筆開始}
\maketitle
%\begin{titlepage}
%\vspace{24pt}
%\textbf{\large 理論と実習の両面から学ぶ}
%\begin{center}
%\includegraphics[width=0.95\linewidth,keepaspectratio]{title1.eps}
%\end{center}
%\vspace{120pt}
%\begin{center}
%\includegraphics[width=0.75\linewidth,keepaspectratio]{title2.eps}
%\end{center}
%\vspace{120pt}
%\begin{flushright}
%\textbf{\LARGE 達哉ん 著}
%\end{flushright}
%\end{titlepage}
%\newpage

\setlength{\parskip}{3ex plus 0.8ex minus 0.4ex}

\pagenumbering{roman}
\chapter*{はじめに}
気づけば、C言語の教科書を書いてから5年、問題集を書いてから2年が経とうとしている。"弟子"として1から教えた若人も手を離れ、弟子も他の知己も含め、時折尋ねに来た時に応ずるばかりとなった。今は、新たに学んだ日本茶でも淹れてほっと一息つきながら、最近はじめた落語に興じている日々である。

変わったのは何もプライベートばかりではない。時はすべてを連れて行くもので、仕事も住まいも知人も大きく変わった。それは、弟子や知己からの質問も例外ではない。ネットワークやシステムに関する質問が増えたのだ。仕事もまた然り。面白いことには、質問も仕事も、ネットワークは抽象化した理論をお話することが主で、システムは個々の具象を扱うことが多いというのが、共通していることである。

個々のシステムについては、各々をマニュアル的に記すことで満足な仕上がりが達せられる。DNS,LDAP,DHCP,POP,SMTPなど各々のサービスの設定は言うに及ばず、ルータの設定コマンドなども、マニュアルで対応すれば良いところであろう。だが、それらの設定の意味を理解することまでは、マニュアルに求められまい。殊に、汎論の範疇と呼ぶべき部分までマニュアルに記載しているようでは、あまりに冗長となってしまう。そこで、これらネットワークの汎論を統べる役目の一冊を用意したいと考えた。周辺の方への説明の際などに便利でもあろうし、無論、自身の記憶を掘り起こす良い復習の契機にもなろう…こうして、この本を著すこととした。

新たに記すときに悩むのは、その性格付けである。コンピュータネットワークの世界には、大部の聖典もあればよく纏まった入門書もある。同じ方向性で書いたとしても、陳腐なものにしかなりはしない。いかなる性格付けにするか、その悩みに鍵をくれたのは幸いに私を慕ってくれる優秀で賢明なる若き友であった。彼らの根問の中で、私は文字に象られている通りの「人に言を通す」という通信こそが大切であると考えたのである。口伝から始まり現代ではインターネットで多くの通信が行われる。その中で、私は非常に多くの通信方法を経験し、また利用している。趣味で演る落語は口伝が基本だ。生で見る落語は、肉声か、あるいはマイクを通す程度である。一方、レコードやCD、DVDで落語を見ることもある。手紙を丁寧に書いて送るのは実に楽しいもので、時折私書が郵便箱にあるとそれだけでとても嬉しい。一方、友人とのやり取りは電子メールやSNS、チャットアプリなどが主力となっている。これだけ多く「人の言を伝える」ことに接する現代、まずは口伝、それから手紙と俯瞰し、そこからアナログ通信・デジタル通信へと発展させていくのはどうだろうか、と考えた。

といっても、演劇の声の出し方や手紙の書きかたなどを丁寧に書くものではない。落語の稽古風景や郵便制度の話を書いたところで脱線にしかならない。そこで、これらの各分野の経験を元に、「人に言を通す」ことを掬い出して話す、そんなテキストを執筆することとした。歴史に通信を学び、そこから電気通信はどうなっていったのか、どういう必要性で技術が生まれたのか、経験から考察を加えて執筆していこうと思った。自身の学んだ電気通信を、経験に従って解釈しなおし、書き直す。地味であり、新規性も乏しいかも知れない。だが、何より話好きな自分の教え方と書くテキストに一貫性を入れるなら、言を通す、それを置いて他になかろう。オーソドックスかもしれないが、様々な分野の経験が何がしかのシナジーをおこし、説明に彩りを加えてくれればと思う。

本書の導入として、我々人類が物事を伝えるために用いた手段には如何なるものがあるのか、それは現代でどのように生きてあるいは滅びたのかという章を設けた。電気通信と程遠く見えるこれらの伝達には、しかし電気通信の要件を定めるような人類のアイディアが詰まっている。そのアイディアの最も元となる部分には「記録」があり、第I部ではその記録技術や信号処理技術についての最小限を記した。ここで記録された情報をまずは電気的に伝えることを考えるのが第II部である。ラジオ・電話・FAXと言った現代の「デジタル信号」からすると一代前の原理であるが、これを知ることはTCP/IP世界にも関わる低レイヤー技術の理解に十分に役立つと考える。第III部からはいよいよ本丸、コンピュータ・ネットワークへと論を進める。階層化モデルに従った理解と代表的なプロトコルを、各々第III部・第IV部として本書の大部分を用いて解説する。付章的であるが、第V部として、冗長化・セキュリティ・トラブルシューティングなど、構築や管理といったより実践的な情報の基盤となる、最初の一歩を記述した。

各章の記述においては、多少の数学を要求する箇所もあるが、高校程度の数学が取り扱えれば読めることを基本とした。ただ、特に前半の章においては厳密な取り扱いにどうしても数学が必要な部分がある。その部分については、まず厳密な取り扱いをせずに論を展開し、後の節で数学的議論を補うこととした。これらの部分には[補遺]と記しており、読み飛ばしても大局に影響しないよう心がけた。また、汎論に対する実例の紹介という意味付けで、各章で出来うる限り演習問題を出題し、巻末にその解答を記述した。汎論の解説ではあっても、手ずから具体例を解釈することは理解の深化に繋がると信ずる。

最後に、この本・これまでの本や、著者と付き合ってくれたこれまでの友人たち、付き合うこととなるこれからの読者たちにもお礼を申し上げたい。南天竺には赤栴檀という立派な木があり、その周には難莚草という草が蔓延ると聞く。この2つの植物は"有無相持ち"、赤栴檀の降ろす露が難莚草に双無き水となり、難莚草の盛衰は赤栴檀に栄養をもたらすという。友人と自身、読者と著者…携わる人々がそんな関係であることを祈り、この本を捧げる。

\begin{flushright}
桂米朝師の「伝」に畏敬を払いながら \\
達哉ん
\end{flushright}


\setlength{\parskip}{1ex plus 0.0ex minus 0.0ex}
\tableofcontents

\newpage

\setlength{\parskip}{3ex plus 0.8ex minus 0.4ex}
\pagenumbering{arabic}
\setcounter{chapter}{-1}
\chapter{「伝える」歴史}
現代で通信という言葉を聞いた時、連想される多くはインターネットを始めとした様々な電気通信だろう。だが、電気通信ができる以前より、人は様々な意思疎通…通信を行ってきた。本章では、電気通信に至るまでの人の意思疎通の営みを見ていく。これらの営みの中で、人間は通信に必要とされる要件を見出してきた。通信とは何か、何が必要なのか。まずは歴史からその要件を学び取っていこう。

\section{声による伝達}

動物には種によって様々な対話方法がある。蝙蝠なら超音波を使い、鳥ならば鳴き声で対話を行う。動物の鳴き声のモノマネで知られる江戸家小猫氏によれば、人間に最も近いゴリラなどは10種類の鳴き声により様々な対話を行うという。例えば、我々がゴリラの鳴き声と聞く「ウホウホ」という鳴き声は、特に喜ばしい時、幼いゴリラが使うことが多い表現なのだそうだ。我々人間にその差異はなかなかわかりづらいが、しかしながら種ごとの鳴き声による伝達は、我々人間の発話による伝達に同じことなのだという。

\subsection{人間の発話}

自らの声帯により何らかの音を出し、これに意味を持たせて伝達する。我々人間が知る多くの動物の鳴き声にはそれほど多くの種類の伝達はなく、概ね自らの感情の伝達が中心となっているようではあるが、人間の発話による伝達もそれと同様である。生まれてすぐの赤子は、泣くという動作を伴った発話により自身が不快であることを伝える。成長し、感情が分化する(図\ref{fig0_1})につれ、何が不快なのかなどもその泣き方に入ってくるようで、親はその泣き方で状況を察することも少なくない。やがて、言葉を覚え話すようになると、それらの言葉の意味を経験から学ぶようになる。家族から広がるコミュニティの中で、共通した意味の言葉を使うようになり、人との伝達・コミュニケーションを始めていくのである。

\begin{figure}[htbp]
\centering
\includegraphics[width=0.6\linewidth,keepaspectratio,bb=0 0 750 750]{fig/fig0_1.jpg}
\caption{ブリッジスの情緒分化図 日本病児保育学会ブログ記事より引用}\label{fig0_1}
\end{figure}

コミュニティの違いによる差異こそあるものの、我々人間の"鳴き声"は言語を為し、その共通認識により対話、伝達が始まったのである。この言語は感情にとどまらず、具象抽象の垣根を超えて叙事や論証、または歌や言葉遊びと言った様々な知的活動の礎となった。動物の鳴き声による対話と対応した人間の会話は、最も原始的な伝達の方法であるものの、様々に技術が発達した今なおコミュニケーションの基礎を為している。通信の始まりは、このような"発話による伝達"であったといえよう。

\subsection{発話の制約}
しかし、単純な発声による発話では、声が届く範囲にしかその意味が届かない。この時の届く範囲とは、空間的な範囲もあれば時間的な範囲でもある。落語会が行われていた会場に、終わった後に入っても落語を聞くことはできないし、また、同じ時間に遠く離れた場所にいても落語を聞くことはできない。肉声の落語を聞くことができるのは、その時その場所で演者と同じ空気を共有していた者に限られる。逆に、その空気を共有している者の中では、選択的に伝えたり聞いたりするということも難しい。高座の上で演者が下座に何らかの指示を出す時、声を使うとすれば客席の幾人かにはどうしても聞こえてしまう。同じ会場で鼾をかいて眠る客がいたとして、これだけを聞かないというのもなかなか無理がある。

先の鼾の例のような雑音があった場合などは特にそうだが、発話によるコミュニケーションでは聞き取れない・追いつかないと言った問題も発生する。マンツーマンのゆっくりした対話であれば言い直してもらえばすむだけの話であるが、先の落語のような例では、そういうわけにも行かない。音自体が聞こえなかった、音は聞こえたけれども判別がつかなかったと言ったレベルから、言語的な問題…イントネーションの差異、語彙にない単語の使用、そもそも言語が異なる、勘違い…と言った知識や思考の問題まで、発話の時には伝達の誤りが起こりうる。

発話による伝達は、原始的であるがゆえに手軽ではあれど制約や問題も内包している。通信・伝達技術というのは、これらの制約や問題を如何に打破するのかという人間の飽くなき挑戦の成果であるとも言える。

\subsection{口伝という方法}
発話の制約を最も単純に外したのは\textbf{口伝}\index{くでん@口伝}である。現在でも民話や民謡など、様々なものが口伝により伝えられているが、これは人間の記憶を介することによって時間や空間の制約を緩和したものといえよう。最も身近なところでは、伝言がそうであろう。もっとも、伝言ゲームに見られる通り、伝達の誤りという問題は解決されているとは言いがたい。「JPCZによる擾乱の影響で山陰地方を中心に荒天となる」などと専門の用語が入った言を伝言したところで、途中に知らぬ人が入れば伝言がうまく行く可能性は低いだろう。また、伝言には意識的・無意識的な取捨選択もあり、伝達の誤りという観点ではむしろ増えているかもしれない。

しかしながら、人間以外の資源が必要ないこの口伝という方法は、最も初期から存在すると同時に、現代まで続いている伝達方法である。口伝の国内における最も古い例としては、稗田阿礼が挙げられる。古事記には彼(彼女?)の誦により伝えられたものが筆録されている。その後の歴史にも、説法や講釈と言った例はほぼ口伝であったようだ。一方、現代で行われている口伝の例としては、先にも持ちだした落語を例に挙げたい。昭和の爆笑王、桂枝雀師は著書「枝雀とヨメはんと七人の弟子」の中で次のように書いておられる。

"一番最初は「ちょっとやるから聞いてなさい」と言って、一字一句口移しで教えることから出発いたします。やらせてみて、ちょっと違うとイントネーションから直す。それが段々お稽古やっていくうちに、五分なら五分、流れを教えるようになる。"

これは、平成のはじめ頃に書かれた文であるし、枝雀師は99年にお隠れになったから、いま他の噺家が同様のやり方をしていると言い切ることはできない。だが、筆者が、この枝雀師の三番弟子、文之助師に"つる"を教わった時は、ほとんどこの通りの教え方であった。ただ、カルチャースクールでのことであるから、録音し、自分で原稿に起こし、その上で直してもらいながらという指導であった。とはいえ、これも口伝には違いない。多くの伝達法がある中で、わざわざこの原始的な口伝という方法で伝承をするのはなぜだろうか。

それは、口伝によってしか伝えられない、空気や情感、発生、身振り手振りの細かな違いなどを伝えられるというところであろう。現代の技術をもってすれば、双方向で似たような指導を遠隔地間で行うことも可能であるように見える。しかし、それでも微妙な空気感や息の間といったものは伝えきれない。落語という芸能は高座の落語家と客席が一体となって作り上げる空間が生命であるから、なるほど、その部分を伝えるためには口伝を取るよりほかないのであろう。また、落語は時代により変化していくものであるから、口伝という形態での変化は逆に追い風となるのであろう。これは、落語の伝承が「ネタの原稿を伝える」のみでないことを示しているといえる。

現代の視点からすれば口伝は古臭く、非効率な手法に見えることも多い。だが、それを意図して取り込んでいる芸に学べる通り、口伝でこそ伝わることもある。通信・伝達技術の多くは、重要と考えにくいものや伝えづらいものを取捨選択しているのである。逆に、その捨てられたものを拾う価値があるシーンや、そこまでの効率を要求しないシーンがあるからこそ、一見古臭く見える通信・伝達の方法も現代に息づいているのである。

\section{書くことによる記録}

口伝は人間の記憶を媒介にして時間や空間の制約を打ち破った。しかしながら、当然全ての人間が稗田阿礼のような記憶力を持つはずもないし、伝達も記憶も個人の状況に依存することとなる。これを打破するのは日本へと輸入された文字であった。

文字とそれを記録する媒体は、ヒエログリフやパピルスと言った著名な例があるとおり古代エジプト文明の時代にまで遡る。楔形文字や漢字と言った様々な文字が青銅器時代に生まれたのである。日本へ文字が入ってくるのは弥生時代とされ、それからの後、8世紀頃には先の稗田阿礼の言を元にした古事記(図\ref{fig0_2})や風土記、日本書紀と言った書物が発行された。

\begin{figure}[htbp]
\centering
\includegraphics[width=0.6\linewidth,keepaspectratio,bb=0 0 246 350]{fig/fig0_2.jpg}
\caption{真福寺収蔵の国宝・『古事記』 Wikipediaより引用}\label{fig0_2}
\end{figure}

文字の輸入に伴い、離れた場所に文字を書いた媒体を送るという形式での伝送も行われるようになっていった。それらの書物の中には何百年という時を経てなお現代で読むことができるものも多い。これは、文字とその記録媒体が時間と空間の制約を打ち破ったということである。著名な数学者の書簡や日記が見つかることもあれば、伝わるつもり無く詠まれた"この世をば 我が世とぞ思う 望月の 欠けたることも なしと思えば"が現代にまで伝わっている(道長本人は記録に残さなかったが、その場に居合わせた人が日記に記したために現代まで伝わっている)など、文字は言語によって表現可能なものの時間的・空間的制約を打ち破ったのである。汚損や焼失といったリスクこそあるものの、口伝や発話の段階からすれば大きく進歩したと言えるのは明らかであろう。

また、これらの媒体は図画の伝達も可能とした。これは、発話や口伝で伝えられなかったものを伝えられるという利点をもたらした。紙媒体は、文字を媒介に発話を記録すると共に、図画として描けるものも記録する媒体なのである。

\subsection{複写}
書物として残されていく

\subsection{紙媒体が捨てたもの}


\section{遠距離の高速通信:狼煙・手旗信号}


しかし郵便は時間がかかります。ではもっと早く伝えたい場合はどうすればいいかというと、視覚に訴えるものを使えばいいという考えに至りました。

このときに「では煙で通信しよう」となって開発されたのが狼煙です。
また、見える範囲ならば情報をリアルタイムに伝えられる通信として手旗信号といったものも開発されました。

このように口伝から書簡や狼煙、手旗信号に通信の手法を変えることで、遠くに素早く通信できたり、時を越えて通信することができるようになりましたが、言葉にこめられる口調や声の違いといった情報を代わりに失いました。

\section{公開と複製:模写・印刷}

人間は書簡といった文字を記録する媒体を使ううちに、その情報をたくさんの人々に公開するために複製しようと試みるようになりました。

最初は紙に書き写すといった作業だけであったのが、版画で同時に並行して複製ができるようになり、コピー印刷で膨大な量の複製が可能になったりと、時代を重ねるごとに複製の量と質が格段に向上していきました。

\section{通信の秘匿:シーザー暗号・方言での暗号}

ここまでは「広める」通信について考えてきましたが、やはりどうしても人には秘密にしたいことを特定の人に伝えたい場合というときはあるものです。
しかし戦場ではまだ書簡を使って通信していましたが、奪われてしまうといったリスクはどうしても拭いきれませんでした。
このときに情報の秘匿性を高めるために暗号が開発されました。

代表的なものだとシーザー暗号が挙げられます(内容は下記の練習問題を参照)。このような一対一で文字変換を行う暗号を単一換字式暗号と言います。

一方で第二次世界大戦中は薩摩弁が難解であるため暗号として使われたのですが、敵軍の日本人捕虜に傍受した暗号を読ませて平文に変換してしまっていたそうです。
このように暗号としてほとんど意味を成さなかったものも存在します。


\section{記録媒体の変化と増加}

記録媒体というものは時代とともに変化していきます。

ここまでの例だと最初は紙や石版などに文字を書き込むというところから始まり、版画やコピーへと移り変わっていったというようなものです。実際このような移り変わりの中で、記録媒体の数は増加していきました。

では現在使われている記録媒体は昔のどんなものからきたのでしょうか。

\subsection{活動写真という媒体}

活動写真は今の映像の記録媒体の発展に深く関わっている媒体です。
概要としては、観客を呼んで動画を見ながら横で活動弁士と呼ばれる人がその映像について解説する、といったものです。

この活動弁士の音声と活動写真の映像を一緒にしてしまおうと作られたのが映画で、それを映画館などではなく一人で見たいという人のために映画を記録する媒体がビデオテープ、DVD、BDとして開発されました。

このように、活動写真は今の映像記録媒体の発展に大きく役立った存在であると言えます。

\section*{演習問題}
\begin{problems}
\item 歴史的なシーザー暗号では、アルファベットを3文字後ろにずらして(a,d,x,y,zをそれぞれd,g,a,b,cに変えるなど)暗号化していた。1024文字以下1行の半角文字列が入力されるとき、それをシーザー暗号化・復号化するプログラムを作成せよ。
\end{problems}


\part{情報の記録}

\chapter{デジタルデータの記録}

\chapter{アナログデータの記録}

\chapter{情報の圧縮}


\part{電気伝送の方法}

\chapter{電気信号の同期}

\chapter{変調}

\chapter{伝送品質の担保}

\chapter{伝送路の方式}

\chapter{電気通信網の構成}


\part{コンピュータ・ネットワークの基礎}

\chapter{通信プロトコル}

\chapter{コンピュータ・ネットワークの構成}

\chapter{ネットワークアーキテクチャ}

\chapter{ネットワーク機器}

\chapter{Layer1:物理層}

\chapter{Layer2:データリンク層}

\chapter{Layer3:ネットワーク層 (1)IPとアドレス}

\chapter{Layer3:ネットワーク層 (2)ルーティング}

\chapter{Layer4:トランスポート層}

\chapter{Layer5〜7:上位3層}


\part{ネットワーク・プロトコル}

\chapter{ネットワークモデル・再論}

\chapter{telnet}

\chapter{DNS}

\chapter{DHCP}

\chapter{FTP}

\chapter{HTTP}

\chapter{メールに纏わるプロトコル} %SMTP,POP3,IMAP

\chapter{認証に纏わるプロトコル} %Kerberos,LDAP


\part{安定通信網の構築と運用}

\chapter{冗長化}

\chapter{セキュリティ}

\chapter{トラブルシューティング手法の基礎}


%\beginappendix
%\chapter{簡易リファレンス}
%演習問題の解答例・解説を掲載する。

\section*{第0章}
\begin{problems}
\item 解答例1
\item 解答例2
\end{problems}

\section*{第1章}
\begin{problems}
\item 解答例1
\item 解答例2
\item 解答例3
\end{problems}


%\chapter{一部解説の付記}
%\input{AppendixB}

%\chapter{Cに関連したWebサービス}
%\input{AppendixC}

%\chapter*{参考書籍等紹介}
%\input{book}

%\chapter*{おわりに}
%\input{last}

\printindex

\end{document}
